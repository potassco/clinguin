\documentclass[12pt,a4paper]{article}

\usepackage[cm]{fullpage}
\usepackage{amsthm}
\usepackage{amsmath}
\usepackage{amsfonts}
\usepackage{amssymb}
\usepackage{xspace}
\usepackage[english]{babel} % Used for date formatting
\usepackage{fancyhdr}
\usepackage{titling}
\usepackage{url}
\usepackage{csquotes}
\usepackage[
    style=numeric,
    bibstyle = numeric,
    citestyle = numeric,
    sorting = none,
    backend=biber
]{biblatex}
\usepackage{parskip}
\usepackage{graphicx}
\usepackage{subcaption}
\usepackage{hyperref}
\usepackage{listings}
\usepackage[table,xcdraw]{xcolor}
\usepackage{nameref}
\graphicspath{ {./img/} }
\addbibresource{summary.bib}

\usepackage{multirow}
\usepackage{listings}
\usepackage[yyyymmdd]{datetime}

\usepackage{mdframed}
\usepackage{lipsum}

\newmdtheoremenv{theo}{Theorem}

%%%%%%%%%%%%%%%%%%%%%%%%%%%%%%%%%%%%%%%%%
% - Declare some maths operators
%%%%%%%%%%%%%%%%%%%%%%%%%%%%%%%%%%%%%%%%%
\DeclareMathOperator*{\argmax}{argmax}

%%%%%%%%%%%%%%%%%%%%%%%%%%%%%%%%%%%%%%%%%
% - Declare langle and rangle in text
%%%%%%%%%%%%%%%%%%%%%%%%%%%%%%%%%%%%%%%%%
\newcommand{\<}[1]{\guilsinglleft}
\renewcommand{\>}[1]{\guilsinglright}

%%%%%%%%%%%%%%%%%%%%%%%%%%%%%%%%%%%%%%%%%
% - Format date to ''dd.mm.YYYY''
%%%%%%%%%%%%%%%%%%%%%%%%%%%%%%%%%%%%%%%%%
\DefineBibliographyExtras{english}{%
  \protected\def\mkbibdateshort#1#2#3{%
    \iffieldundef{#3}
      {}
      {\mkdayzeros{\thefield{#3}}%
       \iffieldundef{#2}{}{.}}%
    \iffieldundef{#2}
      {}
      {\mkmonthzeros{\thefield{#2}}%
       \iffieldundef{#1}{}{.}}%
    \iffieldbibstring{#1}
      {\bibstring{\thefield{#1}}}
      {\dateeraprintpre{#1}\mkyearzeros{\thefield{#1}}}}%
  \protected\def\mkbibdatelong#1#2#3{%
    \iffieldundef{#3}
      {}
      {\stripzeros{\thefield{#3}}%
       \iffieldundef{#2}{}{\nobreakspace}}%
    \iffieldundef{#2}
      {}
      {\mkbibmonth{\thefield{#2}}%
       \iffieldundef{#1}{}{\space}}%
    \iffieldbibstring{#1}{\bibstring{\thefield{#1}}}{\stripzeros{\thefield{#1}}}}%
}


% - Format other dates to ''YYYY-mm-dd''
\renewcommand{\dateseparator}{-}


%%%%%%%%%%%%%%%%%%%%%%%%%%%%%%%%%%%%%%%%%
% - Define LST-LISTINGS
%%%%%%%%%%%%%%%%%%%%%%%%%%%%%%%%%%%%%%%%%
\definecolor{backcolor}{rgb}{0.95,0.95,0.92}

\lstdefinestyle{mystyle}{
    backgroundcolor=\color{backcolor},
    breaklines=true,
    numbers=left,
    numbersep=5pt,
    basicstyle=\ttfamily\footnotesize,
    captionpos=b
}

\lstset{style=mystyle}

%%%%%%%%%%%%%%%%%%%%%%%%%%%%%%%%%%%%%%%%%%
% This part needs customization from you %
%%%%%%%%%%%%%%%%%%%%%%%%%%%%%%%%%%%%%%%%%%

% your group number your names and matriculation numbers here
\newcommand{\groupnumber}{1}
\newcommand{\name}{Alexander Beiser}
\newcommand{\matriculation}{To: The masses}
\newcommand{\gt}{\ensuremath >} 

%%%%%%%%%%%%%%%%%%%%%%%%%%%%%%%%%%%%%%%%%%
%           End of customization         %
%%%%%%%%%%%%%%%%%%%%%%%%%%%%%%%%%%%%%%%%%%

\newcommand{\projnumber}{1}
\newcommand{\Title}{Summary}
\setlength{\headheight}{15.2pt}
\setlength{\headsep}{20pt}
\setlength{\textheight}{680pt}
\pagestyle{fancy}
\fancyhf{} 
\fancyhead[L]{ClingUIn - Project Document}
\fancyhead[C]{}
\fancyhead[R]{\name}
\renewcommand{\headrulewidth}{0.4pt}
\fancyfoot[C]{\thepage}


\begin{document}
\thispagestyle{empty}
\noindent\framebox[\linewidth]{%
 \begin{minipage}{\linewidth}%
 \hspace*{5pt} \textbf{ClingUIn (Summer 2022)} \hfill University of Potsdam - Potassco
 \hspace*{5pt}\\

 \begin{center}
  {\bf\Large ClingUIn Project Document}
 \end{center}

 \vspace*{5pt}\hspace*{5pt} \today\hfill ClingUIn\hspace*{5pt}
\end{minipage}%
}
\vspace{0.5cm}
%%%%%%%%%%%%%%%%%%%%%%%%%%%%%%%%%%%%%%%%%%%%%%%


%\begin{center}
%\textbf{\name} %please fill the information above

%\matriculation %please fill the information above
%\end{center}
%%%%%%%%%%%%%%%%%%%%%%%%%%%%%%%%%%%%%%%%%%%%%%%


\section{Ideas}

\subsection{Project Summary}

\noindent The \textit{ClingUIn} project can be seen as a language extension to a Logic-Program, which adds the functionality to create a User-Interface (UI). Capability wise the UI shall be capable of dynamic updates, i.e. not just a static display of a stable-model(s), but a program where one can interact with the logic program.\\
An example for this is the Sudoku problem. Using the ClingUIn one is able to generate a Sudoku playing-field, where the field is interactive in a sense, that if one selects one of the possible values a single field may have, the program automatically updates the not selected values and fills them in, if only one possibility exists. This is similar to forward checking for Constraint-Satisfactory-Problems (CSPs).\\

\subsection{Prototype}

\noindent Susana Hahn has written a prototype, which implements the Sudoku example from above. The used programming language is Python, further Clingo libraries (like the Clingo-Object-Relational-Mapper (CLORM)) are and the UI-framework tkinter (https://docs.python.org/3/library/tkinter.html) are used. Syntactically the prototype provides an extension by the following keywords (predicates which are used as keywords/tokens):

\begin{itemize}
    \item \textit{window(\< aattribute-name\> ,,\< aargument-key\>,,\<aargument-value\>))} - Defines the general structure/size/etc. of the displayed field
    \item \textit{widget(\< ttype-name\>,,\<uui-id\>,,\<mmaster-id\>))} - Adds a widget (ui-element/item) to the view. The accepted types are \textit{frame} (container) and \textit{menu} (dropdown menu). The \textit{ui-id} should be unique and the \textit{master-id} represents the enclosing item (for top-level frames this is the \textit{window}).
    \item \textit{geo(\<uui-id\>,,\<uui-layout\>,,\<aargument-key\>,,\<aargument-value\>))} - Used for specifying the exact location of a widget. The \textit{ui-id} is of a widget, the \textit{ui-layout} can be chosen from a tuple of \textit{(grid, place, pack)}, the \textit{argument-key} is used for specifying what one wants to set (e.g. row) and the \textit{argument-value} is the corresponding value.
    \item \textit{config(\<uui-id\>,,\<aargument-key\>,,\<aargument-value\>))} - The \textit{ui-id} is of the corresponding widget, the \textit{argument-key} is the property to set (e.g. font width or some color) and the \textit{argument-value} is the corresponding value to the key.
    \item \textit{opt(\<uui-id\>,,\<vvalue\>,,\<ooriginal-state\>))} - This is used for the dynamic updates of the gui. The \textit{ui-id} is for the corresponding widget. In the prototype the stable matches are computed via the \textit{brave} option, and for each possible value the \textit{original-state} can have, also one \textit{opt} is generated. This is used for computing what values a single sudoku field may have.
\end{itemize}

\noindent In general the program works like this: The program scans the logic-program for the syntactic elements above, and then creates the ui with the elements (function \textit{create\_layout()}, uses cautious reasoning). Then the dropdown-menus (i.e. the Sudoku cells) are filled, with the function \textit{update\_options()} (uses brave reasoning). Here for each such dropdown-menu a callback function is added, which when clicked adds an assumption, that a certain value must be used for the sudoku field. Then the process repeats itself.

\subsection{Architecture Proposals}


\subsubsection{Initial Thoughts}

\noindent The architecture of ClingUIn should be flexible in a sense, that it should be compatible with several UI-Libraries/-frameworks. Further it should be efficient (which is not handled at this point, but e.g. incremental updates can be made instead of full updates).\\
The need for flexibility leads to the conclusion, that the program may not directly construct the \textit{tkinter} (or some other frameworks) window, but must create an internal representation first (See Figure \ref{fig:arch-overview}):


\begin{figure}[ht]
    \begin{center}
    \includegraphics[width=0.7\textwidth]{imgs/overview.png}
    \caption{Informal idea of the workings of the program}
    \label{fig:arch-overview}
    \end{center}
\end{figure}

\noindent Therefore for the construction of the program a few things have to be defined/thought about:
 
\begin{enumerate}
    \item Logic-Program Representation
    \begin{itemize}
        \item Syntax of language extension
        \item Initial starting point \(\rightarrow\) syntax of prototype
        \item In general the syntax should be defined in a way that helps understanding of the program
    \end{itemize}
    \item Processing 
    \begin{itemize}
        \item Produces internal representation from the logic program
        \item Main thing: Efficiency
        \item Other things: Libraries to use (depends on selected internal representation), etc.
    \end{itemize}
    \item Internal Representation
    \begin{itemize}
        \item \textbf{Very important to think about now}: Many different options exist, like \textit{XML}, \textit{Json}, \textit{Class-Hierarchy}, \textit{HTML}, ...
        \item Different representations have different pros/cons:
        \begin{itemize}
            \item XML - Pro: Clearly defined structure, portable format. Cons: Format needs to be parsed again before it can be used (efficieny loss), postprocessing will be pretty complex
            \item Json - Pro: For hierarchical GUI tools postprocessing will not be that much, portable format. Cons: Format needs to be parsed again, for non hierarchical GUI tools postprocessing will be quite complex
            \item Class-Hierarchy - Pro: Efficiency, for hierarchical GUI tools postprocessing will be relatively easy. Cons: Non portable format (one is limited to Python represetation), for non hierarchical GUI tools postprocessing will be quite complex.
            \item HTML - Pro: Direct representation of GUI elements, i.e. one can directly view the html, otherwise similar to XML. Cons: For other views than HTML postprocessing will be pretty complex
        \end{itemize}
    \end{itemize}
    \item Postprocessing
    \begin{itemize}
        \item Depends entirely on the chosen internal representation and chosen GUI-framework
        \item Could be implemented as a \textit{1:1} match between internal representation and GUI-framework, which would lead to a modularization of the ClingUIn
        \item Efficiency will be important here...
    \end{itemize}
    \item Initial GUI-Framework to use (tkinter, web/HTML, other)
\end{enumerate}










\end{document}
